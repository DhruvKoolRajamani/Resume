%-------------------------
% Resume in Latex
% Author : Sourabh Bajaj
% License : MIT
%------------------------

\documentclass[letter,11pt]{article}

\usepackage{latexsym}
\usepackage[empty]{fullpage}
\usepackage{titlesec}
\usepackage{helvet}
\usepackage[usenames,dvipsnames]{color}
\usepackage{verbatim}
\usepackage{enumitem}
\usepackage[pdftex]{hyperref}
\usepackage{fancyhdr}
\usepackage{graphicx}
\usepackage{ragged2e}
\usepackage{hyperref}

\hypersetup{
    colorlinks=true,
    linkcolor=blue,
    filecolor=magenta,      
    urlcolor=cyan,
}
\pagestyle{fancy}
\fancyhf{} % clear all header and footer fields
\fancyfoot{}
\renewcommand{\headrulewidth}{0pt}
\renewcommand{\footrulewidth}{0pt}

% Adjust margins
\addtolength{\oddsidemargin}{-0.375in}
\addtolength{\evensidemargin}{-0.375in}
\addtolength{\textwidth}{1in}
\addtolength{\topmargin}{-.5in}
\addtolength{\textheight}{1.0in}

\urlstyle{same}

\raggedbottom
\raggedright
\setlength{\tabcolsep}{0in}

% Sections formatting
\titleformat{\section}{
  \vspace{-10pt}\scshape\raggedright\Large
}{}{0em}{}[\titlerule \vspace{-3pt}]

%-------------------------
% Custom commands
\newcommand{\resumeItem}[2]{
  \item \small{
    \textbf{#1:}{\vspace{-6pt} \noindent #2}
  }
}

\newcommand{\projectItem}[2]{
  \item \small{
    \textbf{#1: }{#2 \vspace{-2mm}}
  }
}

\newcommand{\resumeSubheading}[4]{
  \vspace{-1pt}\item
    \begin{tabular*}{0.97\textwidth}{l@{\extracolsep{\fill}}r}
      \textbf{#1} & \small{#2} \\
      \textit{\small#3} & \textit{\small #4} \\
    \end{tabular*}\vspace{-5pt}
}

\newcommand{\resumeSubItem}[2]{\resumeItem{#1}{#2}}

\renewcommand{\labelitemii}{$\circ$}

\newcommand{\resumeSubHeadingListStart}{\begin{itemize}[leftmargin=*]}
\newcommand{\resumeSubHeadingListEnd}{\end{itemize}}
\newcommand{\resumeItemListStart}{\begin{itemize}}
\newcommand{\resumeItemListEnd}{\end{itemize}\vspace{-5pt}}

%-------------------------------------------
%%%%%%  CV STARTS HERE  %%%%%%%%%%%%%%%%%%%%%%%%%%%%


\begin{document}

%----------HEADING-----------------
\begin{tabular*}{\textwidth}{l@{\extracolsep{\fill}}r}
  \textbf{\href{http://dhruvkoolrajamani.gitlab.io/}{\Large Dhruv Kool Rajamani}} & Email : \href{mailto:dhruvkoolrajamani@gmail.com}{dhruvkoolrajamani@gmail.com}\\
  \href{http://dhruvkoolrajamani.gitlab.io/}{http://dhruvkoolrajamani.gitlab.io/} & Mobile : +91-8800646969 \\
\end{tabular*}

%-----------EDUCATION-----------------
\bgroup
\def\arraystretch{1.5}
\begin{table}[h]
  \centering
  \resizebox{\textwidth}{!}
   \vspace{1mm} \\ 
      \hline
      \textbf{2015 - 2019} \hspace{2mm} & \textbf{Graduation, BTech.} & Mechatronics Engineering, Manipal Institute of Technology, KA \hspace{2mm} & $8.83^{/10}$ \\
      \textbf{2012 - 2014} & \textbf{AISSCE, CBSE Delhi} \hspace{2mm} & Vasant Valley School, Delhi & $95.00^{\%}$ \\ \hline
    \end{tabular}%
  }
\end{table}
\egroup
\vspace{-4mm}

%-----------INTERESTS------------------
\section{Interests}
  \begin{center}
    Legged Locomotion, Central Pattern Generators, Rehabilitative and Assistive Devices, Nonlinear Control
  \end{center}  
  \vspace{-5mm}
%-----------ACHIEVEMENTS---------------
\section{Achievements}
  % \vspace{2mm}
  \begin{itemize}
    \item Best Rover team from Asia, 8th out of 82 teams globally at the University Rover Challenge 2017, Utah.
    \item Best paper presentation at the iACT-2017 conference.
    \item Offered INSIPIRE scholarship by DST, Government of India for top 1\% score in AISSCE 2014 – Declined
  \end{itemize}
  \vspace{-5mm}
%-----------EXPERIENCE-----------------
\section{Experience}
  \resumeSubHeadingListStart

    \resumeSubheading
      {\href{https://biorob.epfl.ch/}{\color{black}BioRob}}{Dr.Auke Jan Ijspeert, Dr.Hamed Razavi, Jonathan Arreguit}
      {École polytechnique fédérale de Lausanne (EPFL)}{January 2018 - Present}
      \resumeItemListStart
        \resumeItem{Implementation of Walking Controller COMAN Robot(COmpliant HuMANoid Platform)}
          {
            \begin{flushleft}
              \noindent Developed a package using OROCOS RTT and ROS frameworks for simulating experiments on walking and implemented a controller for stepping, walking and active balance for the COMAN Humanoid Robot.
              % \begin{itemize}
              %   \item Developed a simulator using OROCOS and ROS on Gazebo.
              %   \item Interfaced a walking controller developed by Dr. Hamed Razavi and validated results            
              % \end{itemize}
            \end{flushleft}
            \vspace{-2mm}
            \begin{table}[h]
              \centering
              \begin{tabular}{lll}
              \textit {\href {https://gitlab.com/Coman-Packages}{OROCOS and ROS Packages}} \hspace{6mm} & \href{https://www.youtube.com/embed/GKYY119oyVY?ecver=2}{Validation through Comparison} \hspace{6mm} & \href {https://dhruvkoolrajamani.gitlab.io/\#experience}{Videos}
              \end{tabular}
              \end{table}
              \vspace{-2mm}
          }
        \resumeItem{Development of a Neuromechanical framework to study animal locomotion}
          {
            \begin{flushleft}
              Developed a package using ROS along with appropriate analysis tools and controllers for simulating modular tetrapoda models with neuromechanical control algorithms.
            \end{flushleft}
            \begin{center}
              \href{mailto:dhruvkoolrajamani@gmail.com?subject=Access to Swampy &body=Please allow me to access your ROS package on Tetrapod locomotion.}{Click here to gain access to this repository}
            \end{center}
            \textit{This work was supported by the Human Frontier Science Program (HFSP) for the Robotics-Inspired Biology project.}
          }
      \resumeItemListEnd
      \vspace{-4mm}

    \resumeSubheading
      {\href{http://robotics.iitd.ac.in/ARL/}{\color{black}Autonomous Robotics Lab}}{Dr.Sudipto Mukherjee}
      {Indian Institute of Technology, Delhi}{2017 – 2018}
      \resumeItemListStart
        \resumeItem{Development of an Underactuated Flexible Manipulator using Differential Flatness}
        {
          \begin{flushleft}
            Designed a 3-link and 4-link planar manipulator on MATLAB and implemented a nonlinear control theory called \textit{Differential Flatness}, allowing the end effector on the manipulator to follow a desired trajectory with just 2 non-colinear input forces.
          \end{flushleft}
        }
        \begin{center}
          \href{https://www.youtube.com/embed/iwXFnWlh8UY?ecver=2}{Simulation demonstrating the trajectory based control of a 4-link manipulator}
        \end{center}
      \resumeItemListEnd

    % \resumeSubheading
    %   {Robotics Lab}{Dr.Ritwik Chattaraj}
    %   {Manipal Institute of Technology, KA}{2017}
    %   \resumeItemListStart
    %     \resumeItem{Modelling and Control of a 3-link inverted pendulum on a cart}
    %       {
    %         \begin{flushleft}
    %           Analysed the motion of a 3 link inverted pendulum on a moving base and design an optimal nonlinear feedforward controller and AKF based controller.
    %         \end{flushleft}            
    %       }
    %   \resumeItemListEnd

    \resumeSubheading
      {\href{http://www.marsrovermanipal.com/}{\color{black}Mars Rover Manipal}}{Dr.Y S Upadhyaya}
      {Manipal Institute of Technology, KA}{2015 - 2017}
      \resumeItemListStart
        \resumeItem{Development of a Mars Rover Prototype}
          {
            \begin{flushleft}
              Developed a Mars Rover prototype that flaunts a modified rocker-bogie suspension, low pressure balloon tires, custom designed scientific testing mechanism, 6-DOF Robotic Arm and can run autonomously. It can traverse harsh Martian like terrain and steep gradients of approximately 1m height.
              \begin{itemize}
                \item \href{https://www.gadgetsnow.com/tech-news/ranked-8th-in-world-mars-rover-manipal-to-build-two-new-rovers/articleshow/60994044.cms}{Best Rover team from Asia, 8th out of 82 teams at the URC 2017.}
              \end{itemize}
            \end{flushleft}
          }
        \resumeItem{Robotic Arm Lead}
          {
            \begin{flushleft}
              Design a 6 DOF Robotic Arm with a payload of 6kgs to conduct tasks similar to those performed by the Curiosity Rover, such as screwing/unscrewing, drilling, pick and place, etc. along with a self adapting gripper attachment. 
            \end{flushleft}
          }\vspace{-3mm}
          \begin{table}[h]
            \centering
            \begin{tabular}{lll}
            \href{https://youtu.be/IulFNAjDEFo}{URC-2017} \hspace{6mm} & \href {https://youtu.be/6Mr8owXfWoc}{Critical Design Review (2017)} \hspace{6mm} & \href{http://www.marsrovermanipal.com/}{Mars Rover Manipal}
            \end{tabular}
            \end{table}
            \vspace{-2mm}
      \resumeItemListEnd

  \resumeSubHeadingListEnd
  \vspace{-6mm}
%-----------PUBLICATIONS---------------
\section{Publications}
\begin{flushleft}
  \textbf{Rajamani, D. K.}, E. D. Pitchika, K. S. Dhankar, and Y. S. Upadhyaya. "Design and development of a linear jawed gripper for unstructured environments." \href{https://ejournal.manipal.edu/mjst/journal_home.aspx}{Manipal Journal of Science and Technology} 3, no. 1 (June 2018). \href{https://drive.google.com/open?id=0B_zt-Aj8SNpHYUk3MzdKNThuUlJZc2hvMGhBWll1Z1FkWHhR}{[link]}
\end{flushleft}
\vspace{-4mm}
%-----------PRESENTATIONS---------------
\section{Presentations}
\vspace{-4mm}
\begin{table}[h]
  \centering
  \resizebox{\textwidth}{!}{%
  \begin{tabular}{ll}
  \begin{tabular}[c]{@{}l@{}}\textbf{December, 2017}\\ \small{\textit{Bhabha Atomic}}\\ \small{\textit{Research Center}}\end{tabular} \vspace{2mm} & \hspace{4mm} \begin{tabular}[c]{@{}l@{}}\small{\textbf{Rajamani, D. K.}, Pitchika, E. D., Dhankar K. S., Shorewala, S., Bansal, D., \& Upadhyaya,}\\ \small{Y. S.(n.d.). Design Overview of a Planetary Exploration Rover for Unstructured}\\ \small{Terrain. 3rd International and 18th National Conference on Machines \& Mechanisms.}\end{tabular} \\
  \begin{tabular}[c]{@{}l@{}}\textbf{July, 2017}\\ \small{\textit{Manipal Institute of}}\\ \small{\textit{Technology}}\end{tabular} \vspace{2mm} & \hspace{4mm} \begin{tabular}[c]{@{}l@{}}\small{\textbf{Rajamani, D. K.}, Pitchika, E. D., Dhankar K. S., \& Upadhyaya, Y. S. (n.d.). Design and}\\ \small{Development of a Linear Jawed Gripper for Unstructured Environments. International}\\ \small{Conference on Applied Sciences, Engineering \& Technology. (Proceedings in Hard Copy)}\end{tabular} \\
  \begin{tabular}[c]{@{}l@{}}\textbf{March, 2017}\\ \small{\textit{Manipal Institute of}}\\ \small{\textit{Technology}}\end{tabular} \vspace{2mm} & \hspace{4mm} \begin{tabular}[c]{@{}l@{}}\small{\textbf{Rajamani, D. K.}, Upadhyaya, Y. S., \& Dhankar, K. S. (n.d.). A comparative Analysis of }\\ \small{Industrial Grade Parallel Gripper and Linear Grippers. ISAB Industrial Automation} \\ \small{and Control TechEvent Day, ISA Bangalore.}\end{tabular}
  \end{tabular}%
  }
  \end{table}

  \vspace{-6mm}
%--------TECHNICAL SKILLS------------
\section{Technical Skills}
\vspace{-4mm}
\begin{table}[h]
  \centering
  \resizebox{\textwidth}{!}{
  \begin{tabular}{ll}
    \vspace{1mm}
    \textbf{Programming} & \hspace{4mm} C++,  Python, C\#, MATLAB, Simulink, Embedded C, \LaTeX, Arduino, HTML, CSS \\
    \vspace{1mm}
    \textbf{Robotics Software} & \hspace{4mm} ROS, OROCOS, GazeboSim, RViz \\
    \textbf{CAD \& CAM} & \hspace{4mm} ANSYS Mechanical Workbench, ADAMS, Soliworks, CATIA V6, AutoCAD, Blender
  \end{tabular}
  }
  \end{table}

  \vspace{-4mm}
%-----------PROJECTS-----------------
\section{Projects}
  \resumeSubHeadingListStart
    \projectItem{Obstacle detection and Path planning for a mobile autonomous robot using computer vision and fuzzy logic}
      {Implemented a heuristic based fuzzy logic approach for path planning through an unknown environment.}
    \projectItem{Traffic Detection using a Kalman Filter}
      {MATLAB Project to detect moving vehicles in a video feed from a traffic
      camera using Kalman filter and Feature detection.}
    \projectItem{LQR based control of a 3-link Linear Inverted Pendulum model (LIP)}
      {
        Analysed the motion of a 3 link inverted pendulum on a moving base and design an optimal LQR based controller.
      }
  \resumeSubHeadingListEnd


%-------------------------------------------
\end{document}
